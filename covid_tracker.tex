\documentclass[12pt]{article}

\usepackage{hyperref}

\title{COVID-19 Tracker Using APIs from Mexican and American authorities}
\author{
	Le Lorier, Yann\\
	\texttt{A01025977}
	\and
	García, Carlos\\
	\texttt{a01025948}
	\and
	Coeto, Victor\\
	\texttt{mat}
}

\begin{document}
	\maketitle
	\tableofcontents

	\section{Introduction}
	The COVID-19 pandemic is rapidly spreading through the world. Which raises the need to make fast decisions with the help of programs able to represent useful information, in real time. We are proposing a solution to pull the information from mexican authorities, such as INEGI, the Dirección General de Epidemiología, or the new CONACYT website, and display data in a graphical way using a client/server architecture.

	\section{Detailed solution}
	The INEGI website offers information, accessible through APIs, massive downloads of data in multiple formats such as: .json, .xml, and DBase.

	\subsection{What we will do}
	Using a C++ driver that we can use to take information from INEGI, we will have to recalculate the information provided by the website, because according to the Subsecretary of Health López-Gatell, the number of COVID-19 cases provided for a previous day may or may not change. So each time that the client requests information, all of the registries in the INEGI table of cases will have to be re-processed.\\
	Briefly:
	\begin{itemize}
		\item Pull information using C++ driver and API from INEGI-CONACYT or CDC
		\item Process comparative graphics between days in Mexico or USA
		\item Process Creation to calculate
		\item threads to display data on a graph
		\item Client/server architecture (IPC)
	\end{itemize}

	\subsection{Tools used}
	The tools that will be used to complete the data request are listed below:
	\subsubsection{INEGI - CONACYT}
	The new CONACYT site developed \cite{conacyt-dw} to face the pandemic has very useful, real-time information regarding the number of
	\begin{itemize}
		\item Confirmed cases
		\item Suspicious cases
		\item Deaths
	\end{itemize}

	By state or by municipalities in Mexico. The main problem, is that to obtain the information, we must re-download the CSV file provided, and we didn't seem to find an available API for that particular site.\\
	However, these APIs exist for information that was availble before the pandemic was declared: such as the INEGI information on healthcare institutions. 
	\subsubsection{CDC}
	CDC provides a well documented, explained environment to exploit data using Socrata Open data API \cite{devsocrata}, which allows to access public data from governments, NGOs, and non-profits around the world. 

	\begin{thebibliography}{99}
		\bibitem{CDC} Centers For Disease and Control APIs \url{https://www.cdc.gov/apis.html}
		\bibitem{INEGI} INegi, 2020 \url{https://datos.covid-19.conacyt.mx/#DOView}
		\bibitem{conacyt-dw} \url{https://datos.covid-19.conacyt.mx/#DownZCSV}
		\bibitem{devsocrata} \url{https://dev.socrata.com}
	\end{thebibliography}


\end{document}